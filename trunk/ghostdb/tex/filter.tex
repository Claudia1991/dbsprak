\chapter{Warum Filter ?}

Das System ``GhostDB'' hat das Problem, dass alle Berechnungen die Bezug auf die unsichtbaren Daten auf dem ``SmartUSB''-Stick nehmen in diesem berechnet werden m�ssen. Da der Speicher f�r die Ausf�hrung auf diesem Ger�t aber begrenzt ist, entsteht hier ein Ressourcenproblem, da alle Daten in den Speicher geladen werden m�ssen und weiterhin in diesem auch Platz f�r das Ergebnis vorhanden sein muss. Die Autoren der Paper haben daher eine M�glichkeit gesucht die Menge an Informationen, die in den Speicher geladen werden m�ssen, um eine Antwort auf die gestellte Anfrage zu erhalten, zu verringern. Das Ergebnis waren die Post-, Pre- und Crossfilter. In dieser Arbeit werden wir nur auf die ersten beiden eingehen.

\section{Prefiltering}

Die Idee beim Prefiltering ist, dass alle Selektionen zuerst ausgef�hrt werden um die gesamte Datenmenge die bei Joins und Mergeoperationen verwendet werden so gering wie m�glich zu halten. Weiterhin wird versucht so viele Selektionen wie m�glich auf dem unsicheren Bereich auszuf�hren, da die Ressourcen hier nicht so beschr�nkt sind, wie im sicheren Bereich. Selektionen auf Daten die den sicheren Teil der ``GhostDB'' betreffen k�nnen allerdings auch nur in diesem ausgef�hrt werden. F�r die Selektionen werden nach M�glichkeit ``Climbing Indices''  verwendet. Die danach folgenden Joins werden dann im sicheren Bereich ausgef�hrt, hierbei wird dann meist ein ``Subtree Key Table'' verwendet.

\section{Postfiltering}

Beim Postfiltering wird ein anderes Verfahren verwendet. Es wird hier zuerst der Join auf dem sicheren Teil der ``GhostDB'' ausgef�hrt. Im folgenden werden die m�glichen Selektionen des unsicheren Bereiches zu den erzeugten Ergebnissen hinzugef�gt. Dies geschieht entweder durch das Erzeugen eines Blommfilters aus diesen Daten, der dann f�r die Filterung der bisherigen Ergebnisse benutzt wird oder genau umgekehrt. Im umgedrehten Fall wird der Bloomfilter aus den vorhandenen Resultaten gebildet und die Daten der Selektion werden mit diesem gefiltert. Die �brig gebliebenen Daten werden dann weiterverabeitet. 
Cross-Filtering kann teilweise bessere Ergebnisse erzielen. Es wird versucht die Kardinalit�t von Zwischenergebnissen des unsicheren Ger�ts so klein wie m�glich zu halten, indem ``Filter'' oder ``Merge''-Operationen mit sicheren Daten so fr�h wie m�glich durchgef�hrt werden.

Im Nachfolgenden wollen wir anhand eines Beispieles die Funktionsweise der Filter nachvollziehbar machen.
