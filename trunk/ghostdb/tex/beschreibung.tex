
Alternative1:

In dieser kurzen Arbeit wollen wir anhand eines selbst gewählten Beispiels, die in der Vorlesung behandelten Techniken des Pre- und Postfilterings anschaulich Darstellen. Die theoretischen Grundlagen hierzu basieren auf dem Themenkomplex der sogenannten "GhostDB". \footnote{GhostDB: Hiding Data from Prying Eyes }
\footnote{GhostDB: Querying Visible and Hidden Data Without Leaks}
\footnote{Vortrag: GhostDB: Hiding Data from Prying Eyes (Technology) }

Wir wollen uns in dieser kurzen Arbeit mit dem Ablauf und d 


Alternative 2:

In der Vorlesung Techniken und Konzepte zum Schutz der Privatsphäre haben wir uns mit verschiedenen Techniken und Verfahren zur Anonymisierung von sensitiven Daten und deren Sicherung beschäftigt. Ein hierbei behandeltes Konzept ist die sogenannte "GhostDB". \footnote{GhostDB: Hiding Data from Prying Eyes }
\footnote{GhostDB: Querying Visible and Hidden Data Without Leaks}
\footnote{Vortrag: GhostDB: Hiding Data from Prying Eyes (Technology) }




Das gesamte Ansatz der "GhostDB" basiert auf verschiedenen Verfahren, wobei wir uns in dieser Arbeit auf das Post- und Prefiltering beschränken wollen und dessen Funktionsweise anhand eines selbst gewählten Beispieles
erläutern wollen.
In den entsprechenden Papern zu diesem Themen wurden meist Beispiele aus dem medizinischen Bereich gewählt, mit Ärzten, Patienten und Untersuchen, etc. Wir haben uns nach reiflicher Überlegung für den Bereich des Flugverkehrs entschieden. In diesem Feld existieren viele verschiedene sensible Daten, die je nach Nutzung anders ausfallen. Eine Fluggesellschaft ist z.B. daran interessiert welcher Reisender zu viel Gepäck mitnimmt um ihn beim nächsten Mal ein teureres Ticket zu verkaufen bis hin zum Terroristen der erfahren möchte welcher Pilot ein bestimmtes Flugzeug fliegt und ob bestimmte Personen wie Polizisten mitfliegen.
Die Herausforderung bestand daher erst einmal in der Suche nach dem aus unserer Sicht besten Datenmodell für dieses Szenario, welches nicht zu groß ausfällt um den Rahmen dieser arbeit nicht zu sprengen und das möglichst wenig Angriffszenarios zulässt.
Unser Datenmodell sieht wie folgt aus:

