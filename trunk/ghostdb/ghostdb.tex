\documentclass[a4paper, pdftex, notitlepage, parskip]{scrreprt}
%
\usepackage[ngerman]{babel}
\usepackage[T1]{fontenc}
\usepackage{lmodern}
\usepackage[latin1]{inputenc}
\usepackage[pdfborder={0 0 0}, pdfpagelabels={false}, bookmarksnumbered={true}, pdfcreator={}, pdfproducer={}]{hyperref}
\usepackage{rotating}
\usepackage[table]{xcolor}
\usepackage{tabularx}
\usepackage{fancyvrb}
\usepackage{amsmath}
\usepackage{enumerate}
\usepackage{float}
\usepackage{multicol}
%
\begin{document}
\title{\vspace{2cm} \Huge GhostDB - Ein Beispiel \vspace{2cm}}
\author{Martin Minor, David Sander}
\date{\today}
\maketitle
\vspace{2cm}
%
Salperwyck, Anciaux, Benzine, Bouganim, Pucheral und Shasha haben ein System namens ``GhostDB'' entwickelt, dass es erm�glicht private Daten vor Unbefugten zu sch�tzen\cite{ghostdb1,ghostdb2,ghostdb3}. Dazu wird die Datenbank auf zwei Ger�te, eine unsichere Umgebung und einen sicheren ``SmartUSB''-Stick, aufgeteilt. Aufgrund der begrenzten Ressourcen m�ssen spezielle Strukturen und Abarbeitungspl�ne verwendet werden. Dieses Beispiel soll die verwendeten Strukturen und Abl�ufe, ``Subtree Key Table'', ``Climbing Index'', ``Pre-'' und ``Post-Filtering'', verdeutlichen.
%
\tableofcontents
\listoffigures
\listoftables
\pagestyle{headings}
%
\chapter{Datenbank}
\begin{table}[ht]
\centering
\begin{tabular}{|c|c|c|c|c|}
\hline
BuchungsID&Datum&Preis&Flug&Reisender\\
\hline
B01&&&&\\
\hline
B02&&&&\\
\hline
B03&&&&\\
\hline
B04&&&&\\
\hline
B05&&&&\\
\hline
B06&&&&\\
\hline
B07&&&&\\
\hline
B08&&&&\\
\hline
B09&&&&\\
\hline
B10&&&&\\
\hline
B11&&&&\\
\hline
B12&&&&\\
\hline
B13&&&&\\
\hline
B14&&&&\\
\hline
B15&&&&\\
\hline
B16&&&&\\
\hline
B17&&&&\\
\hline
B18&&&&\\
\hline
B19&&&&\\
\hline
B20&&&&\\
\hline
B21&&&&\\
\hline
B22&&&&\\
\hline
B23&&&&\\
\hline
B24&&&&\\
\hline
B25&&&&\\
\hline
\end{tabular}
\caption{Buchungen}
\label{tab:Buchungen}
\end{table}
%
\begin{table}[ht]
\centering
\begin{tabular}{|c|c|c|c|c|c|}
\hline
GepaeckID&Flug&Gewicht&Abmessungen&Abgabe&Reisender\\
\hline
G01&&&&&\\
\hline
G02&&&&&\\
\hline
G03&&&&&\\
\hline
G04&&&&&\\
\hline
G05&&&&&\\
\hline
G06&&&&&\\
\hline
G07&&&&&\\
\hline
G08&&&&&\\
\hline
G09&&&&&\\
\hline
G10&&&&&\\
\hline
G11&&&&&\\
\hline
G12&&&&&\\
\hline
G13&&&&&\\
\hline
G14&&&&&\\
\hline
G15&&&&&\\
\hline
G16&&&&&\\
\hline
G17&&&&&\\
\hline
G18&&&&&\\
\hline
G19&&&&&\\
\hline
G20&&&&&\\
\hline
G21&&&&&\\
\hline
G22&&&&&\\
\hline
G23&&&&&\\
\hline
G24&&&&&\\
\hline
G25&&&&&\\
\hline
\end{tabular}
\caption{Gepaeck}
\label{tab:Gepaeck}
\end{table}
%
\begin{table}[ht]
\centering
\begin{tabular}{|c|c|c|c|c|c|}
\hline
FlugID&Start&Ziel&Datum&Flugzeug&Pilot\\
\hline
F01&&&&&\\
\hline
F02&&&&&\\
\hline
F03&&&&&\\
\hline
F04&&&&&\\
\hline
F05&&&&&\\
\hline
F06&&&&&\\
\hline
F07&&&&&\\
\hline
F08&&&&&\\
\hline
F09&&&&&\\
\hline
F10&&&&&\\
\hline
F11&&&&&\\
\hline
F12&&&&&\\
\hline
F13&&&&&\\
\hline
F14&&&&&\\
\hline
F15&&&&&\\
\hline
\end{tabular}
\caption{Fluege}
\label{tab:Fluege}
\end{table}
%
\begin{sidewaystable}[ht]
\centering
\begin{tabular}{l}
Namen aus: \url{http://en.wikipedia.org/wiki/List_of_aviators}\\
\end{tabular}
\begin{tabular}{|c|c|c|c|c|c|c|}
\hline
ReisenderID&Vorname&Name&Geschlecht&Geburtsdatum&Staatsbuergerschaft&Besonderheiten\\
\hline
R01&Jacqueline&Auriol&F&&&\\
\hline
R02&Mike&Bannister&M&&&\\
\hline
R03&Francis&Chichester&M&&&\\
\hline
R04&Doru&Davidovici&M&&&\\
\hline
R05&Eugene Burton&Ely&M&&&\\
\hline
R06&Charles&Fern&M&&&\\
\hline
R07&Sabiha&G�k�en&F&&&\\
\hline
R08&Ernst&Heinkel&M&&&\\
\hline
R09&Tony&Jannus&M&&&\\
\hline
R10&Algene&Key&M&&&\\
\hline
R11&Ruth Bancroft&Law&F&&&\\
\hline
R12&Marie&Marvingt&F&&&\\
\hline
R13&Charles&Nungesser&M&&&\\
\hline
R14&Ivy May&Pearce&F&&&\\
\hline
R15&Harriet&Quimby&F&&&\\
\hline
\end{tabular}
\caption{Reisende}
\label{tab:Reisende}
\end{sidewaystable}
%
\begin{table}[ht]
\centering
\begin{tabular}{|c|c|c|c|}
\hline
FlugzeugID&Typ&Reichweite&Sitzplaetze\\
\hline
A01&&&\\
\hline
A02&&&\\
\hline
A03&&&\\
\hline
A04&&&\\
\hline
A05&&&\\
\hline
\end{tabular}
\caption{Flugzeuge}
\label{tab:Flugzeuge}
\end{table}
%
\begin{sidewaystable}[ht]
\centering
\begin{tabular}{l}
Namen aus: \url{http://en.wikipedia.org/wiki/List_of_aviators}\\
\end{tabular}
\begin{tabular}{|c|c|c|c|c|c|c|c|}
\hline
PilotID&Vorname&Name&Geschlecht&Fluggesellschaft&Geburtsdatum&Gehalt&Flugstunden\\
\hline
P01&John&Alcock&M&&&&\\
\hline
P02&Richard&Bach&M&&&&\\
\hline
P03&Don&Cameron&M&&&&\\
\hline
P04&Jimmy&Doolittle&M&&&&\\
\hline
P05&Amelia&Earhart&F&&&&\\
\hline
P06&Henri&Farman&M&&&&\\
\hline
P07&Roland&Garros&M&&&&\\
\hline
P08&Hilda&Hewlett&F&&&&\\
\hline
P09&Amy&Johnson&F&&&&\\
\hline
P10&Fred&Key&M&&&&\\
\hline
\end{tabular}
\caption{Piloten}
\label{tab:Piloten}
\end{sidewaystable}
%
\begin{table}[ht]
\centering
\begin{tabular}{|c|c|c|}
\hline
Bezeichnung&Medikation&Symptome\\
\hline
\end{tabular}
\caption{Besonderheiten}
\label{tab:Besonderheiten}
\end{table}

\chapter{Subtree Key Table}
Um Joins zu beschleunigen werden in ``GhostDB'' Join Indizes, so genannte ``Subtree Key Tables'', benutzt. F�r alle Tabellen mit ``Nachfahren'', das hei�t Tabellen, die Fremdschl�ssel enthalten, kann ein ``Subtree Key Table'' erstellt werden. Dieser enth�lt den Prim�rschl�ssel der Tabelle selbst, sowie die dazugeh�rigen Fremdschl�ssel und die Fremdschl�ssel der Fremdschl�ssel etc. Der ``Subtree Key Table'', der zur Tabelle Buchungen geh�rt, hat als Prim�rschl�ssel \textit{BuchungsID}. Darauf wird dieser ``Subtree Key Table'' sortiert. Zu jeder \textit{BuchungsID} wird die passende \textit{FlugID} und \textit{ReisenderID} in der entsprechenden Spalte eingetragen. Zu der soeben bestimmten \textit{FlugID} werden au�erdem die dazugeh�rige \textit{FlugzeugID} und \textit{PilotID} eingetragen. 

Die Verwendung dieser Struktur erm�glicht Joins von der zum ``Subtree Key Table'' geh�rigen Tabelle mit beliebigen Nachfahren in einem Schritt durchzuf�hren, beispielsweise \textit{Buchungen} mit \textit{Piloten}. Die ``Subtree Key Tables'' der Tabellen \textit{Buchungen}, \textit{Gepaeck} und \textit{Fluege} sind in den Tabellen \ref{tab:SKTBuchungen} bis \ref{tab:SKTFluege} dargestellt. Zu den Tabellen \textit{Flugzeuge}, \textit{Piloten} und \textit{Reisende} k�nnen keine ``Subtree Key Tables'' erstellt werden, da diese keine Nachfolger haben.

\begin{table}[htbp]
\centering
\begin{tabular}{|c|c|c|c|c|}
\hline
\bfseries BuchungsID&\bfseries FlugID&\bfseries ReisenderID&\bfseries FlugzeugID&\bfseries PilotID\\
\hline
B01&F13&R02&A03&P07\\
\hline
B02&F09&R06&A03&P02\\
\hline
B03&F06&R01&A01&P03\\
\hline
B04&F09&R15&A03&P02\\
\hline
B05&F02&R13&A01&P02\\
\hline
B06&F13&R03&A03&P07\\
\hline
B07&F10&R11&A02&P06\\
\hline
B08&F15&R07&A01&P10\\
\hline
B09&F13&R09&A03&P07\\
\hline
B10&F08&R04&A05&P07\\
\hline
B11&F15&R03&A01&P10\\
\hline
B12&F06&R02&A01&P03\\
\hline
B13&F06&R14&A01&P03\\
\hline
B14&F06&R09&A01&P03\\
\hline
B15&F01&R10&A05&P03\\
\hline
B16&F14&R08&A02&P08\\
\hline
B17&F03&R12&A02&P04\\
\hline
B18&F15&R13&A01&P10\\
\hline
B19&F15&R05&A01&P10\\
\hline
B20&F05&R04&A04&P10\\
\hline
B21&F05&R11&A04&P10\\
\hline
B22&F09&R01&A03&P02\\
\hline
B23&F14&R10&A02&P08\\
\hline
B24&F08&R03&A05&P07\\
\hline
B25&F09&R08&A03&P02\\
\hline
\end{tabular}
\caption{Subtree Key Table Buchungen}
\label{tab:SKTBuchungen}
\end{table}

\begin{table}[htbp]
\centering
\begin{tabular}{|c|c|c|c|c|}
\hline
\bfseries GepaeckID&\bfseries FlugID&\bfseries ReisenderID&\bfseries FlugzeugID&\bfseries PilotID\\
\hline
G01&F08&R04&A05&P07\\
\hline
G02&F08&R04&A05&P07\\
\hline
G03&F09&R01&A03&P02\\
\hline
G04&F10&R11&A02&P06\\
\hline
G05&F06&R02&A01&P03\\
\hline
G06&F06&R14&A01&P03\\
\hline
G07&F05&R11&A04&P10\\
\hline
G08&F08&R04&A05&P07\\
\hline
G09&F09&R15&A03&P02\\
\hline
G10&F13&R02&A03&P07\\
\hline
G11&F10&R11&A02&P06\\
\hline
G12&F08&R03&A05&P07\\
\hline
G13&F10&R11&A02&P06\\
\hline
G14&F09&R08&A03&P02\\
\hline
G15&F01&R10&A05&P03\\
\hline
G16&F01&R10&A05&P03\\
\hline
G17&F13&R09&A03&P07\\
\hline
G18&F15&R07&A01&P10\\
\hline
G19&F05&R04&A04&P10\\
\hline
G20&F05&R04&A04&P10\\
\hline
G21&F08&R03&A05&P07\\
\hline
G22&F13&R02&A03&P07\\
\hline
G23&F06&R01&A01&P03\\
\hline
G24&F14&R08&A02&P08\\
\hline
G25&F15&R13&A01&P10\\
\hline
\end{tabular}
\caption{Subtree Key Table Gepaeck}
\label{tab:SKTGepaeck}
\end{table}

\begin{table}[htbp]
\centering
\begin{tabular}{|c|c|c|}
\hline
\bfseries FlugID&\bfseries FlugzeugID&\bfseries PilotID\\
\hline
F01&A05&P03\\
\hline
F02&A01&P02\\
\hline
F03&A02&P04\\
\hline
F04&A05&P02\\
\hline
F05&A04&P10\\
\hline
F06&A01&P03\\
\hline
F07&A04&P08\\
\hline
F08&A05&P07\\
\hline
F09&A03&P02\\
\hline
F10&A02&P06\\
\hline
F11&A04&P04\\
\hline
F12&A01&P01\\
\hline
F13&A03&P07\\
\hline
F14&A02&P08\\
\hline
F15&A01&P10\\
\hline
\end{tabular}
\caption{Subtree Key Table Fluege}
\label{tab:SKTFluege}
\end{table}

\chapter{Climbing Index}
In ``GhostDB'' wird eine zweite Datenstruktur zur Beschleunigung von Joins eingesetzt: der ``Climbing Index''. Diese haben die weitere Eigenschaft, dass Selektionen beschleunigt werden k�nnen. Zu jedem Attribut jeder Tabelle kann ein ``Climbing Index'' angelegt werden. Dieser enth�lt den Wert des Attributes und eine Liste von IDs zu jeder Tabelle, die ein Vorfahr ist. Betrachten wir beispielsweise das Attribut \textit{Geburtsdatum} der Tabelle \textit{Piloten}. Es existieren so viele Tupel, wie es verschiedene Werte, das hei�t Geburtsdaten, gibt. Zu jedem Geburtsdatum werden 4 Listen zugeordnet. Eine Liste von \textit{PilotIDs}: alle Piloten, die an diesem Tag geboren wurden. Eine Liste von \textit{FlugIDs}: alle Fluege, die von einem Piloten mit diesem Geburtsdatum gesteuert werden. Eine Liste von \textit{BuchungsIDs}: alle Buchungen, die einem Flug zugeordnet sind, der von einem Piloten mit diesem Geburtsdatum gesteuert wird. Eine Liste von \textit{GepaeckIDs}: Eine Liste von Gep�ckst�cken, die einem Flug zugeordnet sind, der von einem Piloten mit diesem Geburtsdatum gesteuert wird.

Die Anwendung eines ``Climbing Index'' erlaubt Selektionen auf dem dazugeh�rigen Attribut und einen anschlie�enden Join mit einer beliebigen Tabelle, die Vorfahr ist, in einem Schritt vorzunehmen. Beispielsweise eine Selektion auf \textit{Pilot.Geburtsdatum} und anschlie�ender Join mit \textit{Buchungen}. Zu den Tabellen \textit{Piloten}, \textit{Flugzeuge}, \textit{Reisende}, \textit{Fluege}, \textit{Gepaeck} und \textit{Buchungen} wurde beispielhaft f�r je ein Attribut ein ``Climbing Index'' erstellt. Diese sind in den Tabllen \ref{tab:ClimbingIndexPilotenGeburtsdatum} bis \ref{tab:ClimbingIndexBuchungenDatum} dargestellt. Prinzipiell k�nnte aber f�r jedes Attribut dieser Tabellen ein Climbing Index erstellt werden. Es k�nnten also 33 ``Climbing Indices'' f�r unsere Datenbank erstellt werden.

\newcolumntype{C}{>{\centering\arraybackslash}X}
\begin{sidewaystable}[htbp]
\centering
\begin{tabularx}{\textwidth}{|C|C|C|C|C|}
\hline
\bfseries Geburtsdatum&\bfseries PilotIDs&\bfseries FlugIDs&\bfseries BuchungsIDs&\bfseries GepaeckIDs\\
\hline
1945-06-04&P10&F05, F15&B08, B11, B18, B19, B20, B21&G07, G18, G19, G20, G25\\
\hline
1948-07-01&P09&&&\\
\hline
1949-11-05&P01&F12&&\\
\hline
1956-07-24&P05&&&\\
\hline
1959-10-06&P07&F08, F13&B01, B06, B09, B10, B24&G01, G02, G08, G10, G12, G17, G21, G22\\
\hline
1960-12-27&P03&F01, F06&B03, B12, B13, B14, B15&G05, G06, G15, G16, G23\\
\hline
1964-12-14&P04&F03, F11&B17&\\
\hline
1967-02-17&P08&F07, F14&B16, B23&G24\\
\hline
1974-05-26&P06&F10&B07&G04, G11, G13\\
\hline
1978-06-23&P02&F02, F04, F09&B02, B04, B05, B22, B25&G03, G09, G14\\
\hline
\end{tabularx}
\caption{Climbing Index Piloten.Geburtsdatum}
\label{tab:ClimbingIndexPilotenGeburtsdatum}
\end{sidewaystable}
%
\begin{sidewaystable}[htbp]
\centering
\begin{tabularx}{\textwidth}{|C|C|C|C|C|}
\hline
\bfseries Typ&\bfseries FlugzeugIDs&\bfseries FlugIDs&\bfseries BuchungsIDs&\bfseries GepaeckIDs\\
\hline
Airbus A300B4&A01&F02, F06, F12, F15&B03, B05, B08, B11, B12, B13, B14, B18, B19&G05, G06, G18, G23, G25\\
\hline
Airbus A340-200&A02&F03, F10, F14&B07, B16, B17, B23&G04, G11, G13, G24\\
\hline
Boeing 747-200B&A03&F09, F13&B01, B02, B04, B06, B09, B22, B25&G03, G09, G10, G14, G17, G22\\
\hline
Bombardier CRJ100 LR&A04&F05, F07, F11&B20, B21&G07, G19, G20\\
\hline
Bombardier CRJ1000&A05&F01, F04, F08&B10, B15, B24&G01, G02, G08, G12, G15, G16, G21\\
\hline
\end{tabularx}
\caption{Climbing Index Flugzeuge.Typ}
\label{tab:ClimbingIndexFlugzeugeTyp}
\end{sidewaystable}
%
\begin{sidewaystable}[htbp]
\centering
\begin{tabularx}{\textwidth}{|C|C|C|C|}
\hline
\bfseries Geschlecht&\bfseries ReisenderIDs&\bfseries BuchungsIDs&\bfseries GepaeckIDs\\
\hline
F&R01, R07, R11, R12, R14, R15&B03, B04, B07, B08, B13, B17, B21, B22&G03, G04, G06, G07, G09, G11, G13, G18, G23\\
\hline
M&R02, R03, R04, R05, R06, R08, R09, R10, R13&B01, B02, B05, B06, B09, B10, B11, B12, B14, B15, B16, B18, B19, B20, B23, B24, B25&G01, G02, G05, G08, G10, G12, G14, G15, G16, G17, G19, G20, G21, G22, G24, G25\\
\hline
\end{tabularx}
\caption{Climbing Index Reisende.Geschlecht}
\label{tab:ClimbingIndexReisendeGeschlecht}
\end{sidewaystable}
%
\begin{table}[htbp]
\centering
\begin{tabular}{|c|c|c|c|}
\hline
\bfseries Ziel&\bfseries FlugIDs&\bfseries BuchungsIDs&\bfseries GepaeckIDs\\
\hline
Antwerpen (ANR)&F01&B15&G15, G16\\
\hline
Barcelona (BCN)&F02&B05&\\
\hline
Chicago (ORD)&F03&B17&\\
\hline
Dortmund (DTM)&F04&&\\
\hline
Erfurt (ERF)&F05&B20, B21&G07, G19, G20\\
\hline
Florenz (FLR)&F06&B03, B12, B13, B14&G05, G06, G23\\
\hline
Genf (GVA)&F07&&\\
\hline
Hamburg (HAM)&F08&B10, B24&G01, G02, G08, G12, G21\\
\hline
Istanbul (IST)&F09&B02, B04, B22, B25&G03, G09, G14\\
\hline
Jakarta (CGK)&F10&B07&G04, G11, G13\\
\hline
Kiew (KBP)&F11&&\\
\hline
London (LHR)&F12&&\\
\hline
Madrid (MAD)&F13&B01, B06, B09&G10, G17, G22\\
\hline
New York (JFK)&F14&B16, B23&G24\\
\hline
Oslo (OSL)&F15&B08, B11, B18, B19&G18, G25\\
\hline
\end{tabular}
\caption{Climbing Index Fluege.Ziel}
\label{tab:ClimbingIndexFluegeZiel}
\end{table}
%
\begin{table}[htbp]
\centering
\begin{tabular}{|c|c|}
\hline
\bfseries Gewicht&\bfseries GepaeckIDs\\
\hline
10&G02, G21\\
\hline
11&G11\\
\hline
12&G15\\
\hline
14&G01, G05, G07, G14\\
\hline
15&G03, G10\\
\hline
16&G04, G25\\
\hline
17&G17\\
\hline
18&G24\\
\hline
20&G08, G09\\
\hline
21&G06, G12, G16, G18\\
\hline
22&G19, G22\\
\hline
23&G13\\
\hline
25&G20, G23\\
\hline
\end{tabular}
\caption{Climbing Index Gepaeck.Gewicht}
\label{tab:ClimbingIndexGepaeckGewicht}
\end{table}
%
\begin{table}[htbp]
\centering
\begin{tabular}{|c|c|}
\hline
\bfseries Datum&\bfseries BuchungsIDs\\
\hline
2010-12-08&B01\\
\hline
2011-01-04&B02\\
\hline
2011-01-23&B03\\
\hline
2011-03-15&B04\\
\hline
2011-04-04&B05\\
\hline
2011-05-13&B06\\
\hline
2011-05-18&B07, B08\\
\hline
2011-05-20&B09\\
\hline
2011-06-04&B10\\
\hline
2011-06-15&B11\\
\hline
2011-06-22&B12, B13, B14\\
\hline
2011-07-01&B15\\
\hline
2011-07-09&B16\\
\hline
2011-07-14&B17\\
\hline
2011-07-26&B18\\
\hline
2011-08-05&B19\\
\hline
2011-08-09&B20, B21\\
\hline
2011-08-10&B22\\
\hline
2011-08-13&B23\\
\hline
2011-08-22&B24\\
\hline
2011-08-23&B25\\
\hline
\end{tabular}
\caption{Climbing Index Buchungen.Datum}
\label{tab:ClimbingIndexBuchungenDatum}
\end{table}

\chapter{Query Execution Plans}
Die folgenden Beispiele zum Erstellen eines "Query Execution Plans" f�r das Pre- und Postfiltering basieren auf der oben schon vorgestellten Datenbasis und der folgenden Anfrage:

\begin{Verbatim}[commandchars=\\\{\}]
SELECT R.ReisenderID, R.Vorname, R.Name, R.Staatsbuergerschaft
FROM Buchungen B, Gepaeck G, Fluege F, Reisende R, Piloten P
WHERE R.Geschlecht='M' AND P.Geburtsdatum<1956-08-01 AND G.Gewicht>20
    AND B.Flug=F.FlugID AND B.Reisender=R.ReisenderID
    AND G.Flug=F.FlugID AND G.Reisender=R.ReisenderID
    AND F.Pilot=P.PilotID
\end{Verbatim}

Die genauen Funktionsaufrufe f�r den gesamten Plan sind jeweils am Ende der Kapitel zu finden.
\section{Prefiltering}

Die Idee beim Prefiltering ist, dass alle Selektionen zuerst ausgef�hrt werden um die gesamte Datenmenge die bei Joins und Mergeoperationen verwendet werden so gering wie m�glich zu halten. Weiterhin wird versucht so viele Selektionen wie m�glich auf dem unsicheren Bereich auszuf�hren, da die Ressourcen hier nicht so beschr�nkt sind, wie im sicheren Bereich. Selektionen auf Daten die den sicheren Teil der ``GhostDB'' betreffen k�nnen allerdings auch nur in diesem ausgef�hrt werden. F�r die Selektionen werden nach M�glichkeit ``Climbing Indices'' verwendet. Die danach folgenden Joins werden dann im sicheren Bereich ausgef�hrt, hierbei wird meist ein ``Subtree Key Table'' verwendet.

Es wird mit der Auswahl des Gewichts begonnen. Es sollen alle Gep�ckst�cke gefunden werden, die schwerer als 20kg sind: {G06, G12, G13, G16, G18, G19, G20, G22, G23}. Diese \textit{GepaeckIDs} werden der sicheren Plattform �bergeben. Es folgt die Selektion auf den Geburtsdaten der Piloten. Diese Eigenschaft ist versteckt und muss daher auf der sicheren Plattform ausgef�hrt werden. Es wird ein ``Climbing Index'' verwendet, um \textit{GepaeckIDs} zu erhalten:  \{\{\}, \{G07, G18, G19, G20, G25\}, \{\}, \{\}\}. Die beiden Listen werden mittels Merge zusammengef�hrt: \{G18, G19, G20\}

\begin{figure}[H]
  \centering
  \includegraphics[width=1\linewidth]{img/Pre1}
  \caption{Teilergebnis Pre-Filtering 1}
  \label{fig:pre1}
\end{figure}

Von Piloten aus wird analog der Weg �ber Buchungen Buchungen beschritten. ``Climbing Index'': \{\{\}, \{B08, B11, B18, B19, B20, B21\}, \{\}, \{\}\}, Merge: \{B08, B11, B18, B19, B20, B21\}.

\begin{figure}[H]
  \centering
  \includegraphics[width=1\linewidth]{img/Pre2.pdf}
  \caption{Teilergebnis Pre-Filtering 2}
  \label{fig:pre2}
\end{figure}

Unter Zuhilfenahme der entsprechenden ``Subtree Key Tables'' wird die Verbindung zur Tabelle \textit{Reisende} hergestellt: \{R07, R04, R04\} bzw. \{R07, R03, R13, R05, R04, R11\}. Mittels Merge wird der Durchschnitt beider Listen gebildet: \{R04, R07\}.

\begin{figure}[H]
  \centering
  \includegraphics[width=1\linewidth]{img/Pre3.pdf}
  \caption{Teilergebnis Pre-Filtering 3}
  \label{fig:pre3}
\end{figure}

F�r die entstandene Liste wird ein Bloom-Filter erzeugt. Die noch ausstehende Selektion R.Geschlecht='M' wird vom unsicheren Ger�t ausgef�hrt und das Ergbnis mittels Bloom-Filter gefiltert.

\begin{figure}[H]
  \centering
  \includegraphics[width=1\linewidth]{img/Pre4.pdf}
  \caption{Teilergebnis Pre-Filtering 4}
  \label{fig:pre4}
\end{figure}

Mittels MJoin werden die noch verbleibenden Projektionen ausgef�hrt: \{$\langle$R04,Doru,Davidovici,Rom�nisch$\rangle$\}.

\begin{figure}[H]
  \centering
  \includegraphics[width=1\linewidth]{img/Pre5.pdf}
  \caption{Teilergebnis Pre-Filtering 5}
  \label{fig:pre5}
\end{figure}

F�gt man die einzelnen Schritte zusammen entsteht folgender Ausf�hrungsplan f�r das Prefiltering:

\begin{figure}[H]
  \centering
  \includegraphics[width=1\linewidth]{img/Pre-Filtering.pdf}
  \caption{Pre-Filtering QEP}
  \label{fig:pre}
\end{figure}

\subsection{Pre-Filtering Funktionsaufrufe}

\begin{enumerate}[1]
\item Vis(Q,Gepaeck,G.GepaeckID) = \{G06, G12, G13, G16, G18, G19, G20, G22, G23\}
\item CI(P.Geburtsdatum,<1956-08-01,G.GepaeckID) = \{\{\}, \{G07, G18, G19, G20, G25\}, \{\}, \{\}\}
\item Merge(($\bigcup 2)\cap 1$) = \{G18, G19, G20\}
\item SJoin(3,$\text{SKT}_{\text{Gepaeck}}$,R.ReisenderID) = \{R07, R04, R04\}
\item CI(P.Geburtsdatum,<1956-08-01,B.BuchungsID) = \{\{\}, \{B08, B11, B18, B19, B20, B21\}, \{\}, \{\}\}
\item Merge($\bigcup 5$) = \{B08, B11, B18, B19, B20, B21\}
\item SJoin(6,$\text{SKT}_{\text{Buchungen}}$,R.ReisenderID) = \{R07, R03, R13, R05, R04, R11\}
\item Merge($4\cap 7$) = \{R04, R07\}
\item BuildBF(8) = BF
\item Vis(Q,Reisende,R.ReisenderID) = \{R02, R03, R04, R05, R06, R08, R09, R10, R13\}
\item ProbeBF(BF,10) = \{R04\}
\item Vis(Q,Reisende,$\langle$R.ReisenderID,R.Vorname,R.Name$\rangle$) = \{$\langle$R02,Mike,Bannister$\rangle$, \\
$\langle$R03,Francis,Chichester$\rangle$, $\langle$R04,Doru,Davidovici$\rangle$, $\langle$R05,Eugene Burton,Ely$\rangle$, $\langle$R06,Charles,Fern$\rangle$, $\langle$R08,Ernst,Heinkel$\rangle$, $\langle$R09,Tony,Jannus$\rangle$, $\langle$R10,Algene,Key$\rangle$, $\langle$R13,Charles,Nungesser$\rangle$\}
\item MJoin(12,11,$\langle$R.ReisenderID,R.Staatsbuergerschaft$\rangle$,8) = \{$\langle$R04,Doru,Davidovici,Rom�nisch$\rangle$\}
\end{enumerate}

Das Problem an dieser Strategie ist allerdings, dass bei zu geringer Selektivit�t Selektionen zu wenige Tupel im Vorfeld aussortiert werden und somit ein gro�er Vorteil des Prefilterings verschwindet. Eine andere Alternative, die dem eben angesprochenen Effekt nicht so stark unterliegt ist das Postfiltering.

\section{Postfiltering}

Beim Postfiltering wird ein anderes Verfahren verwendet. Es wird hier zuerst der Join auf dem sicheren Teil der ``GhostDB'' ausgef�hrt. Im folgenden werden die m�glichen Selektionen des unsicheren Bereiches zu den erzeugten Ergebnissen hinzugef�gt. Dies geschieht entweder durch das Erzeugen eines Bloom-Filters aus diesen Daten, der dann f�r die Filterung der bisherigen Ergebnisse benutzt wird oder genau umgekehrt. Im umgedrehten Fall wird der Bloom-Filter aus den vorhandenen Resultaten gebildet und die Daten der Selektion werden mit diesem gefiltert. Die �brig gebliebenen Daten werden dann weiterverabeitet.

Beim Postfiltering werden zuerst alle Selektionen ausgef�hrt und mittels Join zusammengefasst, die auf den sicheren Bereich zugreifen. Erst danach werden durch ``Fuzzy Filtering''\footnote{GhostDB: Hiding Data from Prying Eyes(Technology) Vortrag} die Selektionen des unsicheren Bereiches hinzugef�gt.

F�r unser Beispiel hei�t dies, dass zuerst die Selektion auf dem Geburtsdatum des Piloten ausgef�hrt wird. Da wir mit der PilotID aber nicht ohne Weiteres weiterarbeiten k�nnen, verwenden wir hierf�r den ``Climbing Index'' auf der GepaeckID und f�hren auf diesem Ergebnis eine Vereinigung aus, da wir nur eine Liste von GepaeckIDs f�r die Weiterverwendung benutzen k�nnen.

\begin{figure}[H]
  \centering
  \includegraphics[width=1\linewidth]{img/Post1.pdf}
  \caption{Teilergebnis Post-Filtering 1}
  \label{fig:post1}
\end{figure}

Diese Selektion ist in unserem Beispiel die Einzige die auf den unsicheren Bereich zugreift. Nach dem Verfahren des Postfilterings werden diese IDs nun f�r einen Join benutzt. Als Ergebnis brauchen wir wiederum ReisendeIDs, da diese f�r die sp�tere Projektion notwendig sind. Um diese IDs zu erhalten bedienen wir uns wieder beim ``Subtree Key Table'' und Filtern uns somit alle ReisendeIDs heraus. Bei den Selektionen aus dem unsicheren Berich erhalten wir aber GepaeckIDs, daher gibt uns der Join auf unsere Anforderung hin eine Menge von Tupeln aus GepaeckID und dazu passender ReisendeID heraus.

\begin{figure}[H]
  \centering
  \includegraphics[width=1\textwidth]{img/Post2.pdf}
  \caption{Teilergebnis Post-Filtering 2}
  \label{fig:post2}
\end{figure}

Im n�chsten Schritt kann nun die Selektion auf dem Gep�ck, welches ein Gewicht gr��er als 20 Kilogramm hat geschehen. Um dies umzusetzen wird das ``Fuzzy Filtering'' benutzt. In unserem Fall ist es so definiert, das ein Bloomfilter �ber eine bestimmte Menge von Daten generiert und dann auf eine andere Menge angewendet wird. Man k�nnte diesen Filter eventuell auf das Ergebnis des Joins beziehen, da diese Ergebnisse aber im sicheren Bereich sind w�rden wir hier nur unn�tig Ressourcen verbrauchen. Eine bessere L�sung ist es den Bloomfilter auf der Selektion der Daten des unsicheren Bereiches zu generieren, da hier mehr Resosurcen zur Verf�gung stehen.

Es wird also ein Bloomfilter auf dem Ergebnis von G.Gewicht>20 gebildet und auf folgende Daten des Joins angewendet:\{$\langle$G07,R11$\rangle$, $\langle$G18,R07$\rangle$, $\langle$G19,R04$\rangle$, $\langle$G20,R04$\rangle$, $\langle$G22,R02$\rangle$, $\langle$G23,R01$\rangle$\}.

Hierbei ist zu beachten das nur die GepaeckIDs des Resultats betrachtet werden, da nur diese im Bloomfilter vorkommen. Auf der Ausgabe des Bloomfilters wird nachfolgend noch eine Projektion auf die ReisendeIDs vorgenommen, da wir die Tupel aus GepaeckID und ReisendeID nicht mehr f�r die n�chsten Schritte ben�tigen. Wir erhalten folgende ReisendeIDs: \{R07, R04, R04, R02, R01\}

\begin{figure}[H]
  \centering
  \includegraphics[width=1\textwidth]{img/Post3.pdf}
  \caption{Teilergebnis Post-Filtering 3}
  \label{fig:post3}
\end{figure}

Als n�chster logischer Schritt w�rde die Selektion auf dem Geschlecht folgen, da aber durch die ``WHERE-Clause'' B.Reisender=R.ReisenderID ebenfalls ReisendeIDs entstehen, die verglichen werden m�ssen, wird dieser Schritt vorgezogen, da andernfalls die Selektion ein zweites Mal erfolgen m�sste. Diesen Mehraufwand wollen wir uns ersparen. Um die entsprechende ``WHERE-Clause'' zu erf�llen, ben�tigen wir BuchungsIDs, die wir dann in entsprechende ReisendeIDs umwandeln k�nnen. Hier verwenden wir das gleiche Schema wie im Prefiltering. Mittels der Selektion auf dem Geburtsdatum des Piloten erhalten wir durch den ``Climbing Index'' eine Liste von Listen mit BuchungsIDs. Diese muss durch eine Vereinigung zu einer Liste zusammengef�hrt werden.

Im Folgenden wird ein SemiJoin angewendet, der mit Hilfe eines ``Subtree Key Tables'' uns die entsprechenden ReisendeIDs ausgibt, die wir dann weiterverwenden. Als Ausgabe werden folgende ReisendeIDs geliefert: \{R07, R03, R13, R05, R04, R11\}

\begin{figure}[H]
  \centering
  \includegraphics[width=1\textwidth]{img/Post4.pdf}
  \caption{Teilergebnis Post-Filtering 4}
  \label{fig:post4}
\end{figure}

Die beiden Mengen von ReisendeIDs werden nun mittels Durchschnitt miteinander vereint. Aus diesem Resultat wird nachfolgend einen Bloomfilter gebaut. Die Ergebnisse die durch die Selektion des Geschlechtes auf die Reisenden (Reisender.Geschlecht='M') im unsicheren Bereich erhaltenen Daten werden durch diesen Filter geschickt. Die nun erhaltenen IDs werden wie im Prefiltering mittels MJoin zusammengef�gt und als Ergebnis erh�lt man: \{$\langle$R04,Doru,Davidovici,Rom�nisch$\rangle$\}.


\begin{figure}[H]
  \centering
  \includegraphics[width=1\textwidth]{img/Post5.pdf}
  \caption{Teilergebnis Post-Filtering 5}
  \label{fig:post5}
\end{figure}

F�gt man nun wie schon oben alle Teile zusammen erh�lt man den kompletten Ablaufplan f�r unsere Anfrage.


\begin{figure}[H]
  \centering
  \includegraphics[width=1\textwidth]{img/Post-Filtering.pdf}
  \caption{Post-Filtering QEP}
  \label{fig:post}
\end{figure}

\subsection{Post-Filtering Funktionsaufrufe}


\begin{enumerate}[1]
\item CI(P.Geburtsdatum,<1956-08-01,G.GepaeckID) = \{\{\}, \{G07, G18, G19, G20, G25\}, \{\}, \{\}\} 
\item Merge($\bigcup 1$) = \{G07, G18, G19, G20, G22, G23\}
\item SJoin(2,$\text{SKT}_{\text{Gepaeck}}$,$\langle$G.GepaeckID,R.ReisenderID$\rangle$) = \{$\langle$G07,R11$\rangle$, $\langle$G18,R07$\rangle$, $\langle$G19,R04$\rangle$, $\langle$G20,R04$\rangle$, $\langle$G22,R02$\rangle$, $\langle$G23,R01$\rangle$\}
\item Vis(Q,Gepaeck,G.GepaeckID) = \{G06, G12, G13, G16, G18, G19, G20, G22, G23\}
\item BuildBF(4) = BF
\item ProbeBF(BF,3) = \{$\langle$G18,R07$\rangle$, $\langle$G19,R04$\rangle$, $\langle$G20,R04$\rangle$, $\langle$G22,R02$\rangle$, $\langle$G23,R01$\rangle$\}
\item Project(6,R.ReisenderID) = \{R07, R04, R04, R02, R01$\rangle$\}
\item CI(P.Geburtsdatum,<1956-08-01,B.BuchungsID) = \{\{\}, \{B08, B11, B18, B19, B20, B21\}, \{\}, \{\}\}
\item Merge($\bigcup 8$) = \{B08, B11, B18, B19, B20, B21\}
\item SJoin(9,$\text{SKT}_{\text{Buchungen}}$,R.ReisenderID) = \{R07, R03, R13, R05, R04, R11\}
\item Merge($7\cap 10$) = \{R04, R07\}
\item BuildBF(11) = BF
\item Vis(Q,Reisende,R.ReisenderID) = \{R02, R03, R04, R05, R06, R08, R09, R10, R13\}
\item ProbeBF(BF,13) = \{R04\}
\item Vis(Q,Reisende,$\langle$R.ReisenderID,R.Vorname,R.Name$\rangle$) = \{$\langle$R02,Mike,Bannister$\rangle$, \\$\langle$R03,Francis,Chichester$\rangle$, $\langle$R04,Doru,Davidovici$\rangle$, $\langle$R05,Eugene Burton,Ely$\rangle$, $\langle$R06,Charles,Fern$\rangle$, $\langle$R08,Ernst,Heinkel$\rangle$, $\langle$R09,Tony,Jannus$\rangle$, $\langle$R10,Algene,Key$\rangle$, $\langle$R13,Charles,Nungesser$\rangle$\}
\item MJoin(15,14,$\langle$R.ReisenderID,R.Staatsbuergerschaft$\rangle$,11) = \{$\langle$R04,Doru,Davidovici,Rom�nisch$\rangle$\}
\end{enumerate}

\chapter{Quellen}
\section{Tabelle \ref{tab:Fluege}}
Die Zielflugh�fen wurden mithilfe folgender Listen ausgew�hlt:

\url{http://de.wikipedia.org/wiki/Kategorie:Flughafen_\%28Europa\%29}\\
\url{http://de.wikipedia.org/wiki/Liste_der_gr\%C3\%B6\%C3\%9Ften_Verkehrsflugh\%C3\%A4fen}

Informationen zu den Flugh�fen selbst, zum Beipsiel IATA-Codes, wurden diesen Webseiten entnommen:

\url{http://de.wikipedia.org/wiki/Flughafen_Antwerpen}\\
\url{http://de.wikipedia.org/wiki/Flughafen_Barcelona}\\
\url{http://de.wikipedia.org/wiki/O\%E2\%80\%99Hare_International_Airport}\\
\url{http://de.wikipedia.org/wiki/Flughafen_Dortmund}\\
\url{http://de.wikipedia.org/wiki/Flughafen_Erfurt-Weimar}\\
\url{http://de.wikipedia.org/wiki/Flughafen_Florenz}\\
\url{http://de.wikipedia.org/wiki/Flughafen_Genf}\\
\url{http://de.wikipedia.org/wiki/Flughafen_Hamburg}\\
\url{http://de.wikipedia.org/wiki/Flughafen_Istanbul-Atat\%C3\%BCrk}\\
\url{http://de.wikipedia.org/wiki/Flughafen_Jakarta}\\
\url{http://de.wikipedia.org/wiki/Flughafen_Kiew}\\
\url{http://de.wikipedia.org/wiki/London_Heathrow_Airport}\\
\url{http://de.wikipedia.org/wiki/Flughafen_Madrid-Barajas}\\
\url{http://de.wikipedia.org/wiki/John_F._Kennedy_International_Airport}\\
\url{http://de.wikipedia.org/wiki/Flughafen_Oslo-Gardermoen}

Die Entfernungen zwischen Start und Ziel wurden mithilfe dieser Webseiten ermittelt:

\url{http://www.distance-calculator.co.uk/airport-information.php?as=SXF&ad=ANR}\\
\url{http://www.distance-calculator.co.uk/airport-information.php?as=SXF&ad=BCN}\\
\url{http://www.distance-calculator.co.uk/airport-information.php?as=SXF&ad=ORD}\\
\url{http://www.distance-calculator.co.uk/airport-information.php?as=SXF&ad=DTM}\\
\url{http://www.distance-calculator.co.uk/airport-information.php?as=SXF&ad=ERF}\\
\url{http://www.distance-calculator.co.uk/airport-information.php?as=SXF&ad=FLR}\\
\url{http://www.distance-calculator.co.uk/airport-information.php?as=SXF&ad=GVA}\\
\url{http://www.distance-calculator.co.uk/airport-information.php?as=SXF&ad=HAM}\\
\url{http://www.distance-calculator.co.uk/airport-information.php?as=SXF&ad=IST}\\
\url{http://www.distance-calculator.co.uk/airport-information.php?as=SXF&ad=CGK}\\
\url{http://www.distance-calculator.co.uk/airport-information.php?as=SXF&ad=KBP}\\
\url{http://www.distance-calculator.co.uk/airport-information.php?as=SXF&ad=LHR}\\
\url{http://www.distance-calculator.co.uk/airport-information.php?as=SXF&ad=MAD}\\
\url{http://www.distance-calculator.co.uk/airport-information.php?as=SXF&ad=JFK}\\
\url{http://www.distance-calculator.co.uk/airport-information.php?as=SXF&ad=OSL}

\section{Tabelle \ref{tab:Reisende}}
Die Namen wurden dieser Liste entnommen:

\url{http://en.wikipedia.org/wiki/List_of_aviators}

Geschlecht, Tag und Monat des Geburtusdatums (Jahr zuf�llig) und Staatsb�rgerschaft wurden diesen Webseiten entnommen:

\url{http://en.wikipedia.org/wiki/Jacqueline_Auriol}\\
\url{http://en.wikipedia.org/wiki/Mike_Bannister}\\
\url{http://en.wikipedia.org/wiki/Francis_Chichester}\\
\url{http://en.wikipedia.org/wiki/Doru_Davidovici}\\
\url{http://en.wikipedia.org/wiki/Eugene_Burton_Ely}\\
\url{http://en.wikipedia.org/wiki/Charles_Fern}\\
\url{http://en.wikipedia.org/wiki/Sabiha_G\%C3\%B6k\%C3\%A7en}\\
\url{http://en.wikipedia.org/wiki/Ernst_Heinkel}\\
\url{http://en.wikipedia.org/wiki/Tony_Jannus}\\
\url{http://en.wikipedia.org/wiki/The_Flying_Keys}\\
\url{http://en.wikipedia.org/wiki/Ruth_Bancroft_Law}\\
\url{http://en.wikipedia.org/wiki/Marie_Marvingt}\\
\url{http://en.wikipedia.org/wiki/Charles_Nungesser}\\
\url{http://en.wikipedia.org/wiki/Ivy_May_Pearce}\\
\url{http://en.wikipedia.org/wiki/Harriet_Quimby}

\section{Tabelle \ref{tab:Flugzeuge}}
Zur Auswahl der Flugzeuge wurden folgende Listen konsultiert:

\url{http://en.wikipedia.org/wiki/List_of_regional_airliners}\\
\url{http://en.wikipedia.org/wiki/List_of_wide-body_aircraft}\\
\url{http://en.wikipedia.org/wiki/Wide-body_aircraft}

Zur Bestimmung der Reichweite und der Anzahl der Sitzpl�tze wurden folgende Webseiten genutzt:

\url{http://en.wikipedia.org/wiki/Airbus_A300}\\
\url{http://en.wikipedia.org/wiki/Airbus_A340}\\
\url{http://en.wikipedia.org/wiki/Boeing_747}\\
\url{http://en.wikipedia.org/wiki/Bombardier_CRJ100/200}\\
\url{http://en.wikipedia.org/wiki/Bombardier_CRJ700/900/1000}

\section{Tabelle \ref{tab:Piloten}}
Die Namen wurden dieser Liste entnommen:

\url{http://en.wikipedia.org/wiki/List_of_aviators}

Geschlecht, Tag und Monat des Geburtusdatums (Jahr zuf�llig) wurden diesen Webseiten entnommen:

\url{http://en.wikipedia.org/wiki/John_Alcock_(RAF_officer)}\\
\url{http://en.wikipedia.org/wiki/Richard_Bach}\\
\url{http://en.wikipedia.org/wiki/George_Cayley}\\
\url{http://en.wikipedia.org/wiki/Jimmy_Doolittle}\\
\url{http://en.wikipedia.org/wiki/Amelia_Earhart}\\
\url{http://en.wikipedia.org/wiki/Henri_Farman}\\
\url{http://en.wikipedia.org/wiki/Roland_Garros_(aviator)}\\
\url{http://en.wikipedia.org/wiki/Hilda_Hewlett}\\
\url{http://en.wikipedia.org/wiki/Amy_Johnson}\\
\url{http://en.wikipedia.org/wiki/The_Flying_Keys}

\begin{thebibliography}{9}
\bibitem{ghostdb1}
Anciaux, N. and Benzine, M. and Bouganim, L. and Pucheral, P. and Shasha, D. \emph{GhostDB: querying visible and hidden data without leaks}. Proceedings of the 2007 ACM SIGMOD international conference on Management of data, 2007.
\bibitem{ghostdb2}
Salperwyck, C. and Anciaux, N. and Benzine, M. and Bouganim, L. and Pucheral, P. and Shasha, D. \emph{GhostDB: hiding data from prying eyes}. Proceedings of the 33rd international conference on Very large data bases, 2007.
\bibitem{ghostdb3}
Salperwyck, C. and Anciaux, N. and Benzine, M. and Bouganim, L. and Pucheral, P. and Shasha, D. \emph{GhostDB: Hiding Data from Prying Eyes (Technology)}. PowerPoint-Presentation: \url{http://www-smis.inria.fr/GhostDB/GhostDB_techno.ppt}, 2008.
\end{thebibliography}

%
\end{document}
